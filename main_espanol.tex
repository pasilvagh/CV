%% start of file `template.tex'.
%% Copyright 2006-2013 Xavier Danaux (xdanaux@gmail.com).
%
% This work may be distributed and/or modified under the
% conditions of the LaTeX Project Public License version 1.3c,
% available at http://www.latex-project.org/lppl/.


\documentclass[10pt,a4paper,sans]{moderncv}        % possible options include font size ('10pt', '11pt' and '12pt'), paper size ('a4paper', 'letterpaper', 'a5paper', 'legalpaper', 'executivepaper' and 'landscape') and font family ('sans' and 'roman')

% moderncv themes
\moderncvstyle{classic}                             % style options are 'casual' (default), 'classic', 'oldstyle' and 'banking'
\moderncvcolor{red}                               % color options 'blue' (default), 'orange', 'green', 'red', 'purple', 'grey' and 'black'
%\renewcommand{\familydefault}{\sfdefault}         % to set the default font; use '\sfdefault' for the default sans serif font, '\rmdefault' for the default roman one, or any tex font name
%\nopagenumbers{}                                  % uncomment to suppress automatic page numbering for CVs longer than one page


% character encoding
\usepackage[utf8]{inputenc}                       % if you are not using xelatex ou lualatex, replace by the encoding you are using
%\usepackage{CJKutf8}                              % if you need to use CJK to typeset your resume in Chinese, Japanese or Korean

% adjust the page margins
\usepackage[scale=0.85]{geometry}
%\setlength{\hintscolumnwidth}{3cm}                % if you want to change the width of the column with the dates
%\setlength{\makecvtitlenamewidth}{10cm}           % for the 'classic' style, if you want to force the width allocated to your name and avoid line breaks. be careful though, the length is normally calculated to avoid any overlap with your personal info; use this at your own typographical risks...


% personal data
% personal data
\vspace{-1em}
\title{Ingeniería Civil Informática}
\firstname{Paulina}
\familyname{Silva}
\address{\color{black}}{\color{black}Vi\~na del Mar, Chile}
\email{pasilvagh@gmail.com}
\homepage{LinkedIn: https://cl.linkedin.com/pub/paulina-silva/61/115/908}   
\phone[mobile]{+56979010117}
\nopagenumbers{}

\begin{document}
%\begin{CJK*}{UTF8}{gbsn}                          % to typeset your resume in Chinese using CJK
%-----       resume       ---------------------------------------------------------
\makecvtitle

\vspace{-3.5em}
\section{Información Personal}
  \cvcomputer{\textbf{Name}}{Paulina Andrea Silva Ghio}{\textbf{Fecha de Nacimiento}}{21 Agosto, 1989}
  \cvcomputer{\textbf{R.U.T}}{17353242-3}{\textbf{Estado Civil}}{Soltera}
  \cvcomputer{\textbf{Nacionalidad}}{Chilena}{\textbf{Escuela}}{Scuola Italiana Arturo dell'Oro Valpara\'iso (2007).}
  \hspace{0.7em}\textbf{Universidad} Universidad T\'ecnica Federico Santa Mar\'ia, Valpara\'iso. Licenciada en Ciencias (2013). Ingeniería Civil Informática (2015). Magister en Ciencias de la Ingeniería Informática (actualmente).

\vspace{-1em}
\section{Resumen Técnico}
    \cvcomputer{\textbf{Lenguajes}}{C, C++,C\#, Java, Python, Ruby, Bash}{\textbf{IDE}}{Eclipse, Microsoft Visual Studio 2008/2010/2012}
    \cvcomputer{\textbf{Web}}{HTML5, CSS3 and Javascript}{\textbf{Database}}{MySQL, PostgreSQL, SQL Server 2008}
    \cvcomputer{\textbf{Sistemas Operativos}}{Linux, Windows 7/8}{\textbf{Frameworks}}{Ruby on Rails, Angular v1}
    \vspace{1 mm}
    \hspace{2em}\textbf{Intereses} \-\hspace{0.15cm}Software Development, Secure Software Architecture and Design, Distributed Systems, Computer Networking.

\vspace{-0.5em}
\section{Proyectos}
\textbf{TacPat4SS, FONDECYT}, (Marzo 2015 - Octubre 2016).
\begin{itemize}
	\item Area: Ingeniería de Software, Desarrollo de Software, Diseño de Software Seguro y Arquitectura.
	\item Trabajo: ``Software Development Initiatives to Identify and Mitigate Security Threats – A Systematic Mapping".
\end{itemize}
\vspace{3 mm}
\textbf{Realidad Alfa}, Feria de Software, realizada por el Departamento de Informática de la UTFSM (2011 - 2012).
\begin{itemize}
	\item Implementación de un wrapper en \(C\#\) para Reconocimiento de Patrones (NyARToolKit) con XNA 4.0 Framework.
	\item Diseño de la interfaz de usuario del Videojuego.
	\item Creación de cartas para reconocimiento de patrones.
\end{itemize}

\vspace{-0.5em}
\section{Experiencia Laboral}
\cventry{Mar 2017 -- Jul 2017}{Profesor de Programación, para carrera de Ingeniería Comercial}{Universidad Técnica Federico Santa María, Valparaíso, Chile}{}{}{}
\cventry{Ago 2016 -- Dic 2016}{Profesor de Introducción a la Ingeniería, para carrera de Ingenieria Civíl Informática}{Universidad Técnica Federico Santa María, Valparaíso, Chile}{}{}{}
\cventry{Ago 2016 -- Hoy}{Analista de Seguridad y Desarrollador de Software}{Trabajo en conjunto de Toeska (UTFSM) y Gestión de Procesos Industriales (GPI)}{}{}{}
\cventry{Mar 2015 -- Hoy}{Ayudante Investigador}{Proyecto TacPat4SS (Fondecyt \#1140408)}{}{}{}
\cventry{Ene 2014 -- Mar 2014}{Desarrollador de Software (Práctica)}{Mosaq Consultores Ltda,  San Sebasti\'an 2807 Of. 212. Las Condes, Santiago}{}{}{}
\cventry{Ene 2011 -- Mar 2011}{Desarrollador de Software (Práctica)}{Desarrollador de Software en la unidad de Ex-alumnos de la UTFSM}{}{}{}
\cventry{2008 -- 2015}{Ayudante de clases y laboratorio}{Clases: Fundamentos de Informática, Redes de Computadores, Sistemas de Gestión. Laboratorios: Laboratorio de Computaci\'on (DI UTFSM), Laboratorio de Integraci\'on Tecnol\'ogica (DI UTFSM). Para más información revisar página de LinkedIn (URL en la parte superior de este documento)}{}{}{}
\cventry{2008 -- 2010}{Ayudante en \textit{Ciclo de Charlas Inform\'aticas}}{"Departamento de Inform\'atica"}{}{}{}
\cventry{Octubre 2009}{Ayudante}{Conferencia Linux, UTFSM}{}{}{}

\vspace{-0.8em}
\section{Habilidades}
\cvline{\textbf{Idiomas}}{Español (Nativo), Inglés (Certificado con \textbf{TOEFL ITP} 2014), Japonés (Certificado por \textbf{Japanese Language Proficiency Test} 2013, nivel N4), Italiano (Básico)}

%\section{LinkedIn}
%For more information please \href{https://cl.linkedin.com/pub/paulina-silva/61/115/908}{\textbf{click here}}.

\end{document}


%% end of file `template.tex'.

%% start of file `template.tex'.
%% Copyright 2006-2013 Xavier Danaux (xdanaux@gmail.com).
%
% This work may be distributed and/or modified under the
% conditions of the LaTeX Project Public License version 1.3c,
% available at http://www.latex-project.org/lppl/.


